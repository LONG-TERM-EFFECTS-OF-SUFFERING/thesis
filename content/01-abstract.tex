\chapter{Abstract}

YouTube, with 2,530 million monthly users, serves as a primary news source for 35\% of U.S. adults. However, current YouTube analysis tools rely on proprietary, black-box methodologies that limit transparency and reproducibility, creating barriers for academic research requiring verifiable systems.

This thesis proposes to develop a web application to \textbf{process}, \textbf{analyze}, and \textbf{visualize} YouTube data using transparent and reproducible methods. The system comprises three integrated components: a data collection module using the YouTube Data API with natural language processing for query translation, a data processing pipeline for cleaning and normalizing responses, and an interactive visualization layer.

The methodology follows an adapted Scrum framework with iterative two-week sprints. Development includes technology stack selection, system architecture design, API integration with LLM-powered interfaces, database implementation, and comprehensive testing. All services are containerized using Docker for reproducible deployment.

The resulting application will provide a no-cost, transparent alternative to commercial tools, democratizing YouTube data analysis for academic researchers, non-profits, and independent investigators. By providing fully documented methodology and open architecture, this work promotes \textbf{reproducible science} and serves as a \textbf{foundational tool} for research in social media analysis, polarization, and information dissemination.
