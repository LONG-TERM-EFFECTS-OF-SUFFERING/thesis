\section{Problem justification}

The use of social media has resulted in profound changes in the pattern of human interaction in today's contemporary society. The presence of social media has emerged as a phenomenon that places itself as a crucial element in the drastic changes in the way people interact in an era where information technology has undeniable dominance in almost every field of life \parencite{azzaakiyyah2023impact}. As these platform become central to public discourse and information dissemination, the methods used to analyze them face increasing scrutiny. However, much of the existing analysis is conducted using proprietary, "black-box" tools that limit transparency and prevent academic certification. This creates a critical need for \textbf{an application built on a reproducible, auditable, and documented methodology} for a major platform like YouTube, thereby empowering researches and the public with verifiable insights.

\subsection{Academic}

This project directly addresses a critical methodological gap in academic research: the lack of a unified, verifiable system for YouTube data analysis. Its core academic contribution is the promotion of \textbf{reproducibility} and \textbf{methodological transparency}. By providing a \textbf{fully documented methodology within this thesis}, the application's architecture and data pipeline allow for the independent verification of research findings. This application provides a \textbf{foundational tool}, enabling researchers to conduct reproducible analyses and serving as a \textbf{validated base for future work} in more advanced data analysis. This creates a valuable and reusable asset for the academic community, ensuring that future topic-level studies of YouTube can be conducted with higher methodological rigor and transparency.

\subsection{Economic}

The primary economic benefit is providing a \textbf{no-cost, high-value alternative} to expensive social listening tools. Commercial tools like Talkwalker or Meltwater represent a significant financial barrier for academic institutions, non-profits and independent researchers, limiting their ability to conduct large-scale analysis. By developing a \textbf{free tool}, this project \textbf{democratizes access} to these analytical capabilities, providing functionality that is otherwise locked behind costly subscriptions. Furthermore, the application \textbf{automates the entire data pipeline}, from collection and cleaning to analysis and visualization. This automation creates significant efficiency gains, saving countless hours of manual labor that researchers would otherwise spend on data preparation.

\subsection{Social}

Given that a substantial portion of the population now uses YouTube as a primary source for news and information, understanding the discourse on this platform is a matter of public interest. This project contributes to social good offering a \textbf{transparent} tool for analyzing public opinion, which can be used by academic researchers. Unlike proprietary "black box" systems, this tool's \textbf{publicly documented methodology} ensures that its processes are completely \textbf{transparent} and \textbf{auditable}. This fosters greater trust and empowers organizations to monitor topics like misinformation, public health discussions or political sentiment with a \textbf{verifiable} and \textbf{accessible} tool.
