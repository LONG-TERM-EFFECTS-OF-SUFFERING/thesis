\section{Problem justification}

The use of social media has resulted in profound changes in the pattern of human interaction in today's contemporary society. The presence of social media has emerged as a phenomenon that places itself as a crucial element in the drastic changes in the way people interact in an era where information technology has undeniable dominance in almost every field of life \parencite{azzaakiyyah2023impact}. As these platform become central to public discourse and information dissemination, the methods used to analyze them face increasing scrutiny. However, much of the existing analysis is conducted using proprietary, "black-box" tools that limit transparency and prevent academic certification. This creates a critical need for open-source application that offers a reproducible, auditable and documented analysis of a major platform like YouTube, thereby empowering researches and the public with verifiable insights.


\subsection{Academic}

This project directly addresses a critical methodological gap in academic research: the lack of a unified, verifiable and open-source system for YouTube data analysis. Its core academic contribution is the promotion of \textbf{reproducibility} and \textbf{open-science}. By providing a fully documented codebase, the application allows the independent replication and verification of data collection and research findings. This application provides a foundational tool, enabling researchers to conduct reproducible analyses and extend this methodology to future studies. This creates a valuable and reusable asset for the academic community, ensuring that future topic-level studies of YouTube can be conducted with higher methodological rigor and transparency.

\subsection{Economic}

The primary economic benefit is providing a \textbf{no-cost}, \textbf{open-source} alternative social listening tools. Commercial tools like Talkwalker or Meltwater represent a significant financial barrier for academic institutions, non-profits and independent researchers, limiting their ability to conduct large-scale analysis. By developing a free tool, this project democratizes access to analytical capabilities. Furthermore, the application \textbf{automates the entire data pipeline}, from collection and cleaning to analysis and visualization. This automation creates significant efficiency gains, saving countless hours of manual labor that researchers would otherwise spend on data preparation. This translates into lower operational cost and allows a more agile and responsive research process.

\subsection{Social}

Given that a substantial portion of the population now uses YouTube as a primary source for news and information, understanding the discourse on this platform is a matter of public interest. This project contributes to social good offering a \textbf{transparent} tool for analyzing public opinion, which can be used by academic researchers. Unlike proprietary "black box" systems, this tool's open-source nature ensures that its methodology is completely \textbf{transparent} and \textbf{auditable}. This fosters greater trust and empowers organizations to monitor topics like misinformation, public health discussions or political sentiment with a \textbf{verifiable} and \textbf{accessible} tool.
