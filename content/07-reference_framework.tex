\section{Reference framework}

\subsection{Glossary}

\begin{itemize}
	\item \textbf{Open-science}: according to \textcite{bertram2023open}, the term \enquote{open science} refers to a range of methods, tools, platforms and practices that aim to make scientific research more accessible, transparent, reproducible and reliable. This includes, for example, sharing code, data and research materials, embracing new publishing formats such as registered reports and preprints, pursuing replication studies and reanalyses, optimising statistical approaches to improve evidence assessment and re-evaluating institutional incentives. The ongoing shift towards open science practices is partly due to mounting evidence that studies across disciplines suffer from biases, underpowered designs and irreproducible or non-replicable results. It also stems from a general desire amongst many researchers to reduce hyper-competitivity in science and instead promote collaborative research that benefits science and society.

	\item reproducibility:

	\item Social network:

	\item Social listening:

	\item Sentiment analysis:

\end{itemize}

\subsection{State of art}

\subsubsection{Commercial social listening tools}

\subsubsection{Existing open-source and academic tools}

\subsubsection{Key academic studies}

\subsection{Theoretical framework}

\subsubsection{The open science moevement}

\subsubsection{Web application architecture}

\subsubsection{Technology stack}
