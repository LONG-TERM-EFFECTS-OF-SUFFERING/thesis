\section{Methodology}

This project adopts an adapted Scrum framework to guide the software development process. While Scrum is traditionally designed for team-based development, this thesis leverages specific Scrum artifacts and events that provide particular value for individual academic software projects, especially in maintaining transparency and systematic progress tracking.

The choice of Scrum is motivated by two primary factors: its \textbf{flexibility} through iterative development and its emphasis on \textbf{structured documentation} through artifacts. This aligns well with the exploratory nature of thesis development, where requirements may evolve as the research progresses.

\subsection{Scrum artifacts adapted for this project}

\subsubsection{Product backlog}

The product backlog serves as an emergent, ordered list of what is needed to improve the product and functions as the single source of work for the project \parencite{schwaber2020scrum}. For this thesis, the \textbf{product backlog} will document all planned features, technical requirements, and improvements for the web application. Each  \textbf{product backlog} item will be explicitly linked to the specific thesis objectives it addresses (as defined in \cref{chap:objectives}), ensuring clear traceability between development activities and research goals. This artifact provides transparency in planning and allows for systematic prioritization of development tasks based on their contribution to thesis objectives.

\subsubsection{Sprint backlog}

Composed of the \textbf{sprint goal}, selected \textbf{product backlog} items, and an actionable plan for delivering an increment \parencite{schwaber2020scrum}, the \textbf{sprint backlog}  will serve as the operational plan for each development iteration. For this individual project, sprints will be two weeks periods during which specific functionality will be developed and integrated into the application.

\subsection{Scrum events adapted for this project}

While Scrum events are designed for team collaboration, certain events will be adapted to provide structure and facilitate self-reflection.

\subsubsection{Sprint planning}

At the beginning of each sprint, sprint planning will determine what can be delivered in the upcoming sprint and how that work will be achieved \parencite{schwaber2020scrum}. This will involve selecting \textbf{product backlog} items and defining a clear \textbf{sprint goal}.

\subsubsection{Sprint retrospective}

The sprint Retrospective provides an opportunity to plan ways to increase quality and effectiveness \parencite{schwaber2020scrum}. For this individual project, retrospectives will be conducted at the end of each sprint to reflect on what went well, what challenges were encountered, and what adjustments should be made for subsequent sprints. This practice supports continuous improvement and systematic documentation of the development process.

Although Scrum is traditionally conceived as a framework for teams, with defined roles including \textbf{Scrum master}, \textbf{product owner}, and \textbf{developers} \parencite{schwaber2020scrum}, this project adapts its core principles to the context of individual thesis development. The emphasis remains on the empirical pillars of \textbf{transparency}, \textbf{inspection}, and \textbf{adaptation} \parencite{schwaber2020scrum}, which are particularly valuable for maintaining rigor and documentation in academic software development projects.

\subsection{Activities to be carried out}

\begin{table}[htbp]
	\centering
	\caption{Activities to be carried out}
	\label{tab:activities}
	\begin{tabularx}{\textwidth}{>{\raggedright\arraybackslash}p{4cm}XX}
		\toprule
		\textbf{Specific objective} & \textbf{Expected result} & \textbf{Activity} \\
		\midrule
		Objective \ref{obj:design}: Design a modular web application architecture. & Complete technical design documentation including system architecture diagrams, component specifications and data models. & Row 1 Col 3 \\
		\midrule
		Objective \ref{obj:implement}: Implement the designed architecture. & Delivery of the complete web application source code implementing all four components with documentation. & Row 2 Col 3 \\
		\midrule
		Objective \ref{obj:test}: Test the implemented system. & Test reports documenting functional and integration testing results for the complete system. & Row 3 Col 3 \\
		\bottomrule
	\end{tabularx}
\end{table}

\subsection{Schedule of activities}
