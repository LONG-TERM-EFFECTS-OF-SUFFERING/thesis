\chapter{Problem statement}

According to Statista \cite{statista2025social}, YouTube is the second most used social network globally, only behind Facebook, with $2,530$ million active monthly users. Notably, 35\% of U.S. adults regularly get news from this video-focused platform \parencite{pewresearch2025social}. These facts highlight how important it is to do any kind of analysis.

Although enterprise social-listening suites (like \cite{talkwalker}; \cite{meltwater}) and social science research tools (like \cite{communalytic}) monitor YouTube, their pipelines and models are proprietary, limiting methodological transparency, reproducibility, and academic scrutiny for topic-level studies. This proprietary approach contradicts with the principles of \textbf{transparent and reproducible science}.

\textbf{Academic researchers} often lack a verifiable system that automates collection, cleaning, descriptive analytics, and insight generation while exposing schemas for independent replication.

This need is exemplified by ongoing research projects that require reliable social media data for analysis. The PROMUEVA project (Computational Models of Social Networks Applied to Polarization in Valle del Cauca) at Universidad del Valle represents such a case. This multidisciplinary project aims to develop computational models to analyze, measure, and predict polarization in social networks, a phenomenon that has significantly impacted Valle del Cauca and Cali specifically \parencite{promueva_project}. To achieve its objectives of developing mathematical models of social network characteristics and user cognitive biases, PROMUEVA requires access to transparent, reliable, and reproducible social media data collection systems. However, the project faces the same methodological barriers: proprietary tools provide data without transparency, while building custom systems demands substantial technical resources. This exemplifies the broader need for accessible, documented tools that enable rigorous academic research on social media phenomena.

\section{Problem formulation}

How to develop a web application to process, analyze and visualize YouTube data?
